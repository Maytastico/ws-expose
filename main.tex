\documentclass[12pt,a4paper]{article}

\usepackage[includeheadfoot,margin=2.5cm]{geometry}

\usepackage{times}
\usepackage[utf8]{inputenc}
\usepackage{listings}

\usepackage{csquotes}
\usepackage{abbrevs}
\usepackage[authordate,bibencoding=auto,strict,noibid,backend=biber]{biblatex-chicago}
\bibliography{Bibliography}

% change language settings here "ngerman", "english"
\usepackage[english,ngerman]{babel}

\title{\titlename}

\author{ Maylis Grune, Muhamedbaqir Al-Rumeil, Niklas Schmidt}
%\author{ \authorid\\ \scriptsize \address }


\date{\exposedate}


\begin{document}
\selectlanguage{ngerman}
\newabbrev{\authorid}{Student Id}
\newabbrev{\authorname}{Vorname Nachname}
\newabbrev{\authormail}{vnachname.mmt-b2015@fh-salzburg.ac.at}
\newabbrev{\exposedate}{14. Dezember 2016}
\newabbrev{\titlename}{Exposé Fermented Reality}
\newabbrev{\supervisor}{Supervisor}
\newabbrev{\address}{FH Salzburg}
\newabbrev{\thesisdate}{Salzburg, Austria, 17.April 2017}


\maketitle

\section*{Einleitung}

Der Klimawandel mitsamt seinen weitreichenden Folgen ist  eine der größten Herausforderungen für die Zukunft. 
Die drastischen Einwirkungen auf Klima, Flora und Fauna \autocite{Consequences:2023}
birgen globale humanitäre und ökonomische Risiken. 
Eine politsch-wirtschaftliche Reaktion auf den Klimawandel ist unerlässlich, 
jedoch darf die Komplexität eines Problems dieser Größenordnung nicht unterschätzt werden. Jeder Sektor einer Marktwirtschaft übt seinen 
Einfluss auf das Klima, hat jedoch seine wirtschaftliche und politische Wichtigkeit. 
Entscheidungsfinder, in einer Demokratie sind das volksgewählte Vertreter, sind also verpflichtet, Prioritäten zu definieren und diese in 
ihren Maßnahmen zur Geltung zu bringen. Politische Maßnahmen sind beispielsweise gesetzliche Regulierungen, Förderungen/Sanktionen oder vergleichbare Instrumente. 

Die Entscheidungen einer Regierung spiegeln also ihre Prioritäten wider, welche transparent und direkt der Öffentlichkeit vermittelt werden müssen, um zukünftige Prioritäten auslegen zu können. 

Es stellt sich also die Frage, welche Entscheidungen die Bundesrepublik Deutschland seit dem Pariser Klimaabkommen 1992 getroffen hat, wie sie motiviert waren, und welche Folgen sie hatten. Dazu werden einzelne Entscheidungen analysiert und anhand von sozialen, wirtschaftlichen und ökologischen Aspekten möglichst neutral bewertet. 
Aufgrund der Vielzahl an politischen Eingriffen, wird der Schwerpunkt auf die “wichtigsten” Entscheidungen gelegt. Kriterien für eine wichtige Entscheidung sind beispielsweise politsche Kontroversen, hohe Kosten oder ein daraus resultierender Bedarf an Umstrukturierung. 

\section*{Inhaltsverzeichnis}
\begin{enumerate}
	\item Problemstellung
	\item Foschungsstand
	\item Wissenslücken
	\item Erkenntnisse
\end{enumerate}

\section*{Literarische Grundlagen}
Zunächst muss ein Überblick zu möglichen Quellen hergestellt werden. 
Hierbei ist es wichtig zu beachten seriöse und vor allem geprüfte Quellen zu verwenden. 
Dabei müssen die zeitlichen Angaben, Kennzahlen und Auswirkungen bestätigt werden. 
Außerdem sollten verschiedene Instrumente der Politik erkannt und eingeordnet werden. 
Statistiken können eventuelle Auswirkungen der Instrumente wie Sanktionen, Förderungen oder 
auch neue Gesetze darstellen. Papers hingegen können diese bereits analysiert zurückgeben, 
Zeitschriften geben eher einen Überblick den Einsatz dieser.

Im fortlaufenden Prozess der Auswertung muss eine Einteilung der Quellen geschehen. 
Hier werden die gesammelten literarischen Quellen in die verschiedenen Aspekte der 
wissenschaftlichen Arbeit eingeordnet: 

\begin{enumerate}
	\item Politische
	\begin{enumerate}
		\item Gesetzestexte
		\item Förderungen durch Institutionen
		\item Sanktionen gegenüber anderen Energiesektoren 
	\end{enumerate}
	\item Wirtschaftliche
	\begin{enumerate}
		\item Statistiken über Wirtschaftswachstum 
		\item Berichte über unübliches Wachstum  
		\item Wissenschaftliche Einschätzungen 
	\end{enumerate}
	\item Sozialwissenschaftliche
	\begin{enumerate}
		\item Befragungen
	\end{enumerate}
\end{enumerate}

Zunächst muss erkannt werden, welche politischen Einflüsse im bereits genannten Zeitraum 
entstanden sind. Dahingehend werden Gesetzestexte und sowie einfache Förderungen analysiert 
und eingeschätzt. Nachdem werden die möglichen Zeiträume der verschiedenen Ereignisse 
festgehalten und durch die anderen Quellen, welche nun hauptsächlich wirtschaftlich und 
sozialwirtschaftlich sind, in Ihrem Einfluss durchleuchtet und beurteilt. 
Somit kann den verschiedenen Ereignissen eine oder auch keine Einflussnahme auf den Erfolg 
für erneuerbare Energien zugeteilt werden. 

\section*{Methodik}
\subsection*{Untersuchungsdesign und Befragungen }
Um den Einfluss der politischen Entscheidungen auf die Akzeptanz und Attraktivität 
erneuerbarer Energien evaluieren zu können werden Umfragen auf zwei Zielgruppen angewandt. 
Zum einen soll repräsentativ für die Wirtschaft Unternehmen aus diversen Wirtschaftsbereichen 
zum anderen Privatleute und junge Erwachsene, welche als repräsentativ für die Zielgruppe der Unternehmen 
gelten befragt werden. Vorgesehen sind Unternehmen aus dem tertiären Sektor 
( Dienstleistungen, Gastronomie, Handel und Bankwesen ) und dem Sekundären Sektor
( Bauindustrie, Wasserversorgung und Handwerk )  zu befragen. 
Die Umfragen sollen dabei im Raum Stuttgart durchgeführt werden. Stuttgart dient dabei als großer 
Wirtschaftsstandort mit einem Anteil von 32 \% aus dem sekundären Sektor und 67\% aus 
dem teritären Sektor \autocite{stuttgart:2023} und bildet damit die gesuchten Wirtschaftsbereiche großflächig ab. 
Zum anderen ist Stuttgart von der geographischen Nähe als Befragungsort geeignet.

Der tertiäre Wirschaftssektor ist unteranderem auch besonders interessant,
da dieser 75\% und er sekundäre Wirschaftssektor 23\% der Erwerbtätigen in Deutschland
abdeckt \autocite{workers:2022}. 
Somit kann ein breites Stimmungsbild aus der Wirtschaft soll dadurch entstehen. 
Ein quantitativer Ansatz wird hier gewählt um messbare Ergebnissen zu erhalten. 
Die Daten sollen hier aufgrund der fehlenden Akzeptanz von online Umfragen, 
bevorzugt Face to Face oder auf telefonischen Wege erhoben werden.
Interessant wären hierbei die Meinung von Unternehmenden und Mittelständern im Alter von 35- 50 Jahren.

Zusätzlich zur Unternehmensbefragung sollen Umfragen der allgemeinen Bevölkerung geführt werden. 
Zielgruppe der Befragungen sollen junge Erwachsene im Alter von 18-35 sein. Darunter würden Schüler, 
Studenten oder Auszubildende und Angestellte fallen. Diese sollen unter anderem das aktuelle Meinungsbild 
der Kunden und Kundinnen der Unternehmen widerspiegeln. Hier wäre besonders interessant was die jüngere Generation 
über die Akzeptanz von erneuerbaren Energien hält und wie sich die Daten im Vergleich zur älteren Generation unterscheiden. 
Die Umfragen sollen hierbei ebenfalls im Raum Stuttgart durchgeführt werden. Dabei werden hier Onlineumfragen durchgeführt. 

Das Resultat soll eine breit aufgestellte Datenbasis darstellen, 
die es ermöglicht die Fragestellung auf fundierten Daten zu unterstützen und zu bekräftigen.  

\subsection*{Inhaltsanalyse}
Um die Attraktivität von erneuerbare Energien messbar zu machen, 
soll nach der Investitionsbereitschaft der Unternehmen und Privatleuten 
gefragt werden. Besonders wichtig wäre wie politische Entscheidungen die 
Entscheidung beeinflussen oder beeinflusst haben. Wie attraktiv werden erneuerbare 
Energien durch politische Förderungen wie die Solarstromvergütung, eine KfW-Förderung 
für Befragten oder Steuerliche Vorteile? 
 Zusätzlich soll ein Meinungsbild abgefragt werden. 
 Hat das Thema für die Befragten Relevanz? Würden die Befragten den 
 Ausbau von erneuerbaren Energien durch die Abnahme von Ökostrom unterstützen? 
 Wie wichtig wäre es einer Privatperson, dass ein Unternehmen seine Produkte mit Strom 
 aus nachhaltigen Quellen herstellt? Würde eine Privatperson ein Unternehmen bevorzugen 
 welches seine Produkte mit nachhaltigen Strom herstellt? Die Umfragen sollen Literatur 
 gegenübergestellt werden. Dabei soll ermittelt werden ob und inwieweit es Korrelationen 
 zu den Umfragen gibt. 

 \subsection*{Zu erwartende Ergebnisse}
 Man kann damit rechnen, dass durch die Veränderung in Politik und Gesellschaft 
 zu Gunsten erneuerbarer Energien ein äußerst positives Stimmungsbild und damit 
 eine Akzeptanz nachhaltigen Energieformen zu erwarten ist. Durchaus kann
 davon ausgegangen werden, dass Privatenleuten das nachhaltige Wirtschaften von Unternehmen
 immer wichtiger wird. Wodurch die Attraktivität erneuerbarer Energien für Unternehmen steigen 
 würde. Politischer Einfluss durch Gesetzesreformen und Förderungen tragen zusätzlich 
 dazu bei, dass erneuerbare Energien akzeptierter werden. 

 

% nocite print the whole bibliography. Remove nocite to
% print only the cited references.
\nocite{*}
\printbibliography


\end{document}
